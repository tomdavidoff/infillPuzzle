\documentclass[12pt]{article}

\usepackage{amsmath}
\usepackage[margin=1in]{geometry}
\usepackage{graphicx}
\usepackage{setspace}
\usepackage{booktabs}

\usepackage[style=authoryear,backend=biber]{biblatex}
\addbibresource{infillPuzzle.bib}

\begin{document}

\title{Where does the infill housing go? Addition versus division}
\author{Thomas Davidoff\thanks{thomas.davidoff@sauder.ubc.ca} and Tsur Somerville\thanks{tsur.somerville@sauder.ubc.ca\\Sauder School of Business, University of British Columbia}\\\textbf{Under continual revision.} Updated version is available at:\\ https://github.com/tomdavidoff/infillPuzzle/blob/main/text/main.pdf}
  \maketitle

\onehalfspacing
\begin{abstract}

	Jurisdictions across Canada and the U.S. have recently allowed
	``missing middle'' infill housing, such as accessory dwelling units and
	multiplexes, in residential zones formerly restricted to single-family
	use. Some studies have documented that infill housing is less prevalent
	in more expensive neighbourhoods. This is puzzling in that the right to
	add density should be more valuable in neighbourhoods where prices are
	higher. To resolve this seeming contradiction, we develop a
	parsimonious model emphasizing that different implementations of
	missing middle zoning feature different mixes of ``addition'' (adding
	to allowable square footage) and ``division'' (subdividing a fixed
	amount of square footage into multiple units). Addition is more
	valuable where price per square foot of built space is greatest.
	Division is more valuable where the elasticity of price in square
	footage is less positive.  Infill housing will thus be less prevalent
	in pricier neighbourhoods when the upzoning features relatively little
	addition and more division if (wealthier) households with greater
	willingness to pay for space  are concentrated in expensive
	price-per-square-foot neighbourhoods. Housing transactions and
	permitting data in upzoned neighbourhoods of Vancouver, Portland, and
	Minneapolis provide mixed evidence on these theoretical contributions.
	Division-only policies generally have take-up spatially negatively
	correlated with price per square foot. Mixed addition-and-division
	reforms have mixed uptake. 

\end{abstract}

\section{Introduction}

As housing affordability has grown in political salience over the last decade,
a growing number of U.S. and Canadian jurisdictions have introduced reforms to
allow ``missing middle'', ``infill'', or ``gentle density'' housing types in
formerly single-family-only residential zones. This housing might take the
form of accessory dwelling units such as Vancouver's laneway homes,
``multiplex'' structures with multiple homes on single lot, townhomes, or
small apartment buildings. As has been emphasized, e.g. by
\textcite{bruecknerSingh} and \textcite{HsiehMoretti},restrictive zoning
causes deadweight loss most where real estate is most valuable. It is thus
natural to expect that the option to densify urban land would be exercised
most commonly in the neighborhoods where structure prices per unit are
highest.

As described in more detail below, several recent empirical studies have
documented that infill housing is, in fact, less prevalent in highly expensive
neighbourhoods. This paper presents a parsimonious model of density choice that
illustrates how ``addition'' and ``division'' effects likely work in opposite
directions with respect to the relationship between infill prevalence and
neighbourhood price levels. To the extent that infill policies allow more
square feet to be built on the same lot (addition), this should lead to greater
take-up where price net of cost per square foot is greatest. However, to the
extent that densification simply allows more units to be created out of the
same square footage (division), then in the likely case that (wealthier)
households with a greater elasticity of willingness to pay for square footage
sort into more expensive single-family neighbourhoods, infill will be less
prevalent there. That positive sorting is theoretically likely: in standard
monocentric models affluent households may suburbanize due to flat bid-rent
curves generated by income-elastic demand for housing,\footnote{See
\textcite{Wheaton77} and earlier work of Muth, Alonso, Mills, and Beckmann
cited therein.} but with single-family zoning leading many households to a
fixed (over-) consumption of lot size, the normality of demand for location
quality should generate positive sorting.

This paper studies patterns of redevelopment in three cities that enacted very
modest, but broad upzonings: Vancouver, BC, in Canada, and Portland,
OR and Minneapolis, MN in the U.S. These cities have enacted different upzonings
with different emphases on addition and division. For each city, we evaluate the
spatial correlation between neighbourhood price levels and the prevalence of
infill housing permits following the different upzoning. In Vancouver, laneway
homes (allowed since 2009) essentially allow an ADU to replace a somewhat
smaller two-car garage. This involves both addition and division (through
sharing of the lot). Takeup has been biased towards the (relatively!)
lower-priced East Side of the city with convergence recently. Duplexes, allowed
since late 2018 allow \emph{less} total floor residential area than single
family homes with laneway homes added, and not surprisingly, this ``division
only'' infill form has been taken up predominantly in less expensive East Side
neighbourhoods. Relatively recently-allowed multiplexes, that allow a bit more
density have also been concentrated on the East Side; a quantitative exercise
does not make a clear prediction of the siting given the offetting addition and
division effects.

In Minneapolis, three different infill zones have been created. We find that
the zone relatively close to downtown, in which more density has been
permitted, is the only of the three that exhibits a positive correlation
between infill prevalence and neighbourhood price levels. In Portland, OR,
where the upzoning has been more uniform across the city, we find a negative
correlation between infill prevalence and neighbourhood price levels. However,
the sorting is more mixed in the areas with stronger uptake. In all three
cities, we find a highly positive correlation between neighbourhood average
hedonically-adjusted price per square foot levels and the elasticity of price
with respect to home square footage.

\section{A model of infill housing}

To see the offsetting addition and division effects formally, consider a
developer who owns a lot of size 1 unit of land area. Profits $\pi$ are equal
to square feet $f$ built\footnote{equivalently floor-area-ratio or
floor-space-ratio because of the lot size assumption.} times revenue per square
foot minus cost per square foot. Zoning regulations impose caps on $f$ on the
number of square feet built, and $n$ the number of units built. Price per
square foot is a function of two numbers: $s\equiv \frac{f}{n}$, the number of
square feet per unit, and $\theta$ an increasing index of location quality. We
assume the following:

\begin{itemize}
	\item $f$ and $n$ are continuous and subject to sufficiently small changes to enable calculus.
	\item Price is a function only of $s$ and $\theta$. 
	\item $\frac{\partial^{2} p}{\partial s \partial \theta}$ is
		a meaningful quantity. A fuller model would explicitly
		approximate sorting of heterogeneous households into locations,
		so that price would be a function of the distribution of
		households bidding on homes, itself dependent on the
		distribution of housing characteristics across
		locations.\footnote{A relatively parsimonious version of such a
		model is in \textcite{TervioMaattanen}.} Given positive sorting
		and demand for space, we assume $\frac{\partial^{2} p}{\partial
		s \partial \theta} > 0$.
	\item Cost per square foot is constant in $\theta$. In fact, finishes are likely to be more expensive and higher-end in high-quality locations conditional on the product defined by $f$ and $n$. In empirical application, $p-c$ should be adjusted for this factor. 
\end{itemize}

\begin{equation}
  \pi(f,n) = f\left[p\left(f/n,\theta\right) - c(f,n)\right]
\end{equation}

The benefit from incrementing floor area $f$ is the ``addition effect'':
\begin{equation}
  \label{eq:mb}
  \frac{\partial \pi}{\partial f} = \left[p\left(n,\theta\right) - c\right] + f\left[\frac{\partial p}{\partial s}\frac{1}{n} - \frac{\partial c}{\partial f}\right]
\end{equation}

The derivative of this benefit with respect to the quality index $\theta$ is:
\begin{equation}
	\label{eq:mbtheta}
	\frac{d\frac{\partial \pi}{\partial f}}{d\theta} = \underbrace{\frac{\partial p}{\partial \theta}}_{\text{addition effect, positive}} + \underbrace{\frac{1}{n}\frac{\partial^2 p}{\partial s \partial \theta}.}_{\text{inverse division effect, likely positive}}
\end{equation}

The benefit of increasing $n$ (and hence reducing $s=\frac{f}{n}$) is:
\begin{equation}
	\label{eq:mbn}
	\frac{\partial \pi}{\partial n} =  - \left[s^{2}\frac{\partial p}{\partial s} + \frac{\partial c}{\partial n}\right].
\end{equation}

The effect of $\theta$ on this marginal benefit is given by the negative of the (likely positive) cross partial of price with respect to space and quality.:
 
\begin{equation}
	\label{eq:mbntheta}
	\frac{d\frac{\partial \pi}{\partial n}}{d\theta} = \underbrace{-s^{2}\frac{\partial^2 p}{\partial n \partial \theta}}_{\text{division effect: likley negative}}.
\end{equation}


\subsection{Main theoretical contribution}
\begin{itemize}
  \item \emph{Addition effect:} Infill will be more attractive in high-priced neighborhoods when policies are focussed purely on allowing additional square footage. 
  \item \emph{Division effect:} Where infill policies emphasize splitting the same square footage into more units, if sorting is such that $\frac{\partial^2 p}{\partial s \partial \theta} > 0$, infill will be less attractive where $\theta$ (and price) is greater.
\end{itemize}

\section{Related Literature}

Redevelopment more broadly has been long studied in the literature, beginning with the dynamic models of urban growth such as \textcite{Brueckner_1981}, \textcite{HochmanPines_1980}, and \textcite{Wheaton_1982}. Those papers relied on perfect foresight and had no redevelopment frictions caused by local land use policies. Land use regulation has typically been identified by economists as the primary cause of limits on redevelopment and housing supply more broadly that increases  housing prices. \textcite{GyourkoMolloy2015} and \textcite{Molloy2020} provide comprehensive reviews of  the literature that looks at this relationship between land use regulation and housing supply.

\textbf{\textcite{MarantzElmendorfKim2023_ADUReforms} find that ADU placement in California is widespread, but concentrated in convenient, but low- and moderate-rent neighbourhoods, not in more expensive neighbourhoods. Similar findings are in \textcite{BruecknerThomaz2024} and
study accessory dwelling unit (ADU) permitting in California cities and find that ADU permitting rates are lower in higher-priced neighbourhoods, even after controlling for income and other factors. Huang et al (2023)  analyze the construction of duplexes and triplexes in Seattle following a 2019 upzoning reform and find that these housing types are more likely to be built in lower-priced neighbourhoods. \textcite{DongHansz_2019} examine the impact of Portland's Residential Infill Project (RIP) on housing construction patterns and find that the new zoning has led to more infill development in lower-priced areas. These findings suggest that the relationship between housing prices and infill development is more complex than initially anticipated.}


More recently, and in the wake of broad rezoning occurring in cities around the world, a number of studies have  attempted to estimate the effects of large scale upzonings on new supply. These studies either estimate the effect of a single broad based introduction of higher density or compare blocks or other small areas that have received higher allowed density to those that have not received them in the same jurisdiction. In the former, identification comes from comparing sites treated with an increase in allowed density to those whose density remains unchanged, while in the later identification comes from the rolling introduction of higher density generating multiple treatments.

\textcite{Anagol_etal2021} analyze a broad-based 2016 zoning reform in S\~{a}o Paulo, Brazil that allowed up to a 1.4 unit increase the maximum allowed built area ratio (BAR) - analogous to floor area ratio (FAR) or floor space ratio (FSR) - for over 45,000 blocks. Over the reform, the average BAR increase was more modest, from 1.54 to 2.09. Over six years they note a 66 percent increase in multi-family permit applications for treated blocks compared to the controls (or .012 permits per block per year) resulting in an aggregate 1.9 percent increase in the stock of housing and a 0.5 percent decline in prices.
2For the effects of land use regulation constraints on supply and affordability see Molloy 2020.

Like S\~{a}o Paulo, Auckland, NZ also introduced a single broad upzoning. Implemented by the Auckland Council (the government body administering the metro area) in late 2016, Auckland Unitary Plan (AUP), increased the allowed number of units and density on about 75 percent of the metro area's residential land. That reform, through mostly moderate increases in density (about 90 percent of the upzoned area allowed for three units per parcel, with the remaining 10 percent at higher densities) is estimated to have increased the potential developable stock of housing by 300 percent (see the discussion in \textcite{Greenaway2024}). \textcite{GreenawayPhillips2023} estimate the impact of the AUP upzoning on new housing construction. They demonstrate that the large increase in new construction (equal to an estimated 4.1 percent of the existing housing stock over five years) following the introduction of the AUP did not result from units that would otherwise have been built or built in the areas that were not rezoned. The increase was in more intensive (higher capital, attached) housing in the core areas of the region. To estimate the likely long run effects of the AUP, \textcite{Greenaway2025} estimates the changes to land price gradients in a monocentric model and then estimates the changes in housing stock implied by the land price changes. He argues that the land price changes are consistent with a long run 23 percent increase in the area's housing stock.

A concern with the removal of density constraints is that the increase in the development option value can increase house prices without increasing supply, as the hurdle to exercise the option rises (\textcite{Clap_etal2012}). \textcite{Greenaway_etal2021} estimate the short-run effect of the AUP upzoning on house prices while attempting to control for these redevelopment premiums. They find that upzoning increases the value of the redevelopment premium, but that this positive effect is declining in magnitude with the degree of existing development density, and negative for properties with already high density. In terms of the effects on affordability of the AUP, Greenaway-McGrevy 2023 uses a synthetic control approach to estimate that rents were 15- 35 percent less than they would be otherwise, depending on unit type.

The above papers examine the effects of rezoning within a five-year period. \textcite{BuchlerLutz2024}  are able to examine longer-term effects studying Swiss zoning reforms that increased densities over a multi-decade period starting in the 1990s. They take advantage of the rolling introduction of increases in the allowed maximum density across the Canton of Zurich from 1995 to 2020 to address concerns about endogenous choice of locations for upzoning. Their paper uses an event study approach for each of the 100m x 100m cells the create for Zurich, with the event date varying as the zoning reforms occur over the 25-year period, Not surprisingly, they find larger effects than did \textcite{Anagol_etal2021}, with a 9 percent increase in living space and housing units in up-zoned compared to not up-zoned areas over a five-to-ten-year window. The range of changes and locations allows them to identify heterogeneity in the response to the zoning change based on the size of the increase in allowed density and the extent of whether buildings are already constrained by the initial zoning.

Two studies address cities in our sample. \textcite{Dong2024a} assesses the Portland, OR upzoning that we also study. He finds that as a result of these changes, ``moderate middle'' housing increased from 13.4 percent to 44.7 percent of new building permits in single family zones, but does not provide an estimate of the total changing in housing.\textcite{GuMunro_2025} use synthetic controls to test the effects of the Minnesota reforms, finding lower prices and rents, but surprisingly no change in permits.

While not the entire city, New York's series of neighbourhood-level rezoning analyzed by \textcite{Liao2023} affected approximately 40 percent of the city between 2004 and 2013, with upzoning in many areas. Using changes in FAR and comparing areas in either side of the boundary between changed and unchanged areas, \textcite{Liao2023} finds an overall 4 percent increase in housing supply over seven years, with this increase at 8 percent for areas that were treated with the largest increases in FAR. Using the same zoning reform, Peng 2023 creates a spatial model with worker residential mobility and finds as a counterfactual to the observed outcomes a lower effect of a 0.7 percent long-run increase in the stock. In the spirit of models of redevelopment, \textcite{Rollet_2025} draws from the same set of zoning changes. He creates a dynamic model of demand for floor space and interact this with zoning reforms in New York City to assess the effects of zoning relaxations, on redevelopment. He finds that the large fixed costs of redevelopment, especially in dense areas, limits the short-run effects of rezoning, so that increases in supply take considerable time.

Another city that implemented scattered but still broad upzoning was Seattle. The Mandatory Housing Affordability (MHA) reform, which occurred mainly in 2019, upzoned 33 neighbourhoods in the city. However, along with allowing increased density, the MHA imposed inclusionary zoning, requiring a percentage of the new units in each development to be set aside for lower income households at lower cost. \textcite{KrimmelWang2024} study the effects of the inclusionary zoning requirement on rezoning uptake. Using within neighbourhood difference-in-differences analysis, they find a differentially larger increase in census blocks not upzoned with the inclusionary zoning restrictions. This effect is most severe for lower density (four story and below) developments.

While the work cited above examines upzoning in individual cities, \textcite{Stacy_etal2023} create a cross-city panel to study the effects of relaxing zoning restrictions on new construction more broadly. They conduct newspaper text searches for approximately 180 land use regulatory changes between 2000 and 2019 in eight US metro areas. About half (98) of these lessen restrictions through either allowing ADUs (35), increasing FARs (15). If they observe a relaxation in any measure, they observe The discrete characteristic of having loosened restrictions is associated with a 0.8 percent increase in residential postal addresses in the jurisdictions that loosen restrictions, when compared to jurisdictions that do not.  They also find the effects come mainly from above-median rent areas.

Here we study the most modest of increases in unit density, the bottom end of the ``missing middle.'' As such our subject areas most similar to the Auckland upzoning. The introduction of zoning changes to allow for accessory dwelling units (ADUs) on existing single family lots is the most common regulatory relaxing found in \textcite{Stacy_etal2023}. Several papers also look at the effects of ADUs as a form of very modest increases in allowed density: \textcite{Davidoff_etal2022} estimate the size of density spillovers on neighbouring properties; \textcite{BruecknerThomaz2024} examine the pattern of uptake of ADUs in Los Angeles, identifying neighbourhood correlates with uptake. In both cases property owners and builders respond to the opportunity to add a smaller unit to a property without reducing the size of the main single family detached structure with an increase in aggregate housing supply. Our work differs in that we examine a case where the existing unit must be torn down, density increases by more (both in units and square feet), and we seek to compare a jurisdiction where redevelopment occurs compared with one where it does not.

Critics of upzoning have argued that it either does not lead to new construction or that the new construction that occurs does not approve affordability.\footnote{See \textcite{Been_etal2019} for a summary of increased housing supply as the central policy to address worsening with housing affordability and responses to the critics of increased density.} This literature focuses on much more significant changes than those we study. An example is the work by \textcite{Freemark2020} in Chicago of the effects of upzonings in 2013 and 2015  that increased FARs, heights, and allowed units close to rail transit stations. He finds increases in values (up to 23 percent) and permits in the treated areas when compared with controls, though the latter effects is only statistically different than zero at a 10 percent test).

With undeveloped lots, an alternative to allowing duplexes would be halving lot sizes. Using data from Atlanta to calibrate a model, \textcite{Ma_2025} estimates the effect of this type of forced upzoning. She finds that given that new homes are large units built for higher income users, that forcing  upzoning results in improvements in affordability as when a single large unit cannot be built, more smaller units are constructed instead. \textcite{vonBergmannLauster} find large empirical gains in rare cases of lot-splitting in Vancouver.

\section{Institutional setting}

In this paper, we assess patterns of redevelopment of single-family sites to modestly higher density in Minneapolis, MN; Portland, OR; and Vancouver, BC, Canada. Below, we describe the evolution of the relevant policies and their details in each of these three cities.

\textit{Minneapolis, MN USA}

Minneapolis was moved away from single-family zoning, with the City Council voting in December 2018 to allow more than one unit on all single-family lots. The changes came into effect in January 2020 following approval by the region’s Metropolitan Council in 2019. The rezoning divides the city into three major regions, with additional distinct policies for transit corridors. These corridors were rezoned to allow 3–6 storey buildings, with 10–30 storeys permitted close to transit stations. Additionally, in May 2021, off-street parking minimums were eliminated.

Our interest is in the three largest regions of the city: ``Interior-1,'' ``Interior-2,'' and ``Interior-3,'' where the rezoning permitted up to three units on all previously single-family-zoned lots. The largest land area is in Interior-1, which consists of neighborhoods most distant from the city’s core. In this area, the floor space ratio (FSR) remained unchanged at 0.5, with a maximum height of 2.5 storeys or 28 feet (8.5 meters). In Interior-2, the ring around the core, the same restrictions apply. Only in Interior-3—the single-family areas immediately south and northeast (across the Mississippi River) of the downtown core—did the rezoning increase FSR, from 0.5 to 0.6 for duplexes and 0.7 for triplexes. Triplexes also received a relaxation of the height restriction, to three storeys or 42 feet (12.8 meters). The standard residential lot size in Minneapolis is 5,000 sf, with newer single-family units typically maximizing the allowed buildable area.

\textbf{PREDICTION: likelier positive sorting in Interior-3, where there is more addition, versus others only division}

\textit{Portland, OR USA}

Broad upzoning in Portland was triggered by the Oregon legislature’s passage of House Bill 2001 in July 2019. This bill effectively banned exclusive single-family zoning and required jurisdictions with over 25,000 residents, among other changes, to allow up to four units per lot. Cities were given until June 2022 to update their zoning codes to meet the bill’s requirements.

The City of Portland enacted its first round of reforms, the Residential Infill Project (RIP-1), effective in August 2021. This first of two rounds applied to zones nearer the city centre with smaller lots (land use zones R2.5, R5, and R7).\footnote{These zones reflect allowed lot sizes: R2.5 allows 1 lot per 2,500 sq ft; R5 allows 1 lot per 5,000 sq ft; and R7 allows 1 lot per 7,000 sq ft.} Allowed FSR varied by zone and by the number of units built per lot, with an additional 0.1 FSR for each added unit above the single-family maximum of 0.7, 0.5, and 0.4 in zones R2.5, R5, and R7, respectively.\footnote{For a 2-, 3-, or 4-unit development, the maximum FAR would be 0.8, 0.9, and 1.0 in the R2.5 zone, respectively; 0.6, 0.7, and 0.8 in the R5 zone; and 0.5, 0.6, and 0.7 in the R7 zone.}

The City enacted a second round of reforms (RIP-2) in June 2022. This second round extended the RIP-1 changes to the lower-density, larger-lot zones (R10 and R20, for 10,000 and 20,000 sq ft lots, respectively), which tended to be further from the city’s urban core.\footnote{The shares of single-dwelling land by zone are roughly R2.5 – 7\%, R5 – 41\%, R7 – 18\%, R10 – 18\%, and R20 – 5\%; see \textcite{DongHansz_2019}.} Additional changes under RIP-2 included the ability to subdivide lots and create townhouse-style, side-by-side, fee-simple units across all zones, as well as the extension of accessory dwelling unit (ADU) permissions into the R10 and R20 zones. Overall, RIP-2 extended RIP-1 to the remaining single-family zones, while permitting more flexibility in structure types and options for fee-simple, rather than condominium or strata, ownership.

\textit{Vancouver, BC Canada}

In Vancouver, increases in allowed densities on single-family lots occurred in two rounds. The first was the September 2018 approval of duplexes, with up to two additional rental basement suites (one per unit) permitted on nearly all single-family lots.\footnote{The primary excluded single-family area is the heritage district in the Shaughnessy neighbourhood, comprising very large houses on large lots.} The inclusion of rental basement suites is consistent with longstanding single-family zoning, under which nearly all lots are allowed a single main structure with a rental basement suite and, where a lane is present, a rental infill laneway house or ADU.

Duplexes and single-family units faced the same maximum FSR, between 0.6 and 0.7. These FSR restrictions did not apply to laneway units, which were permitted an additional 0.16 FSR. Duplexes could not include laneway units. Thus, a single-family property could contain one owned unit and two rental units, while a duplex could contain two owned units, each with its own rental suite. Total built FSR could therefore be higher for a single-family property with a laneway unit than for a duplex.

Following the Province of British Columbia’s elimination of exclusive single-family zoning in November 2023, the City of Vancouver extended multiplex permissions to all residential zones. At the same time, the maximum allowed size for ADU units was increased from 0.16 to 0.25 FSR. The number of permitted units varies between three and six (and up to eight for purpose-built rental), depending on lot area and frontage. A standard 33’ × 122’ lot would allow three units. To allow four units, lot area would have to exceed 4,984 sq ft or frontage would have to exceed 43.6’.

With the 2023 changes, maximum allowed FSR depended on several factors. Single-family homes and duplexes were permitted maximum FSRs of 0.6 and 0.70, respectively, with an additional 0.25 FSR allowed for an ADU on single-family lots. For multiplexes (three or more units), the default maximum was 0.7, with 1.0 FSR attainable if one of the following applied: (i) all units are secured-rental housing; (ii) one unit is designated as ``below-market'' homeownership; or (iii) the developer pays a density bonus, which can vary from \$3.00 to \$140.00 per square foot of additional density depending on lot size and location. For lots of sufficient size, 7–8 rental units are also possible at an FSR of 1.0.

\textbf{PREDICTION: duplex, only division should have negative sorting. Laneway ambiguous, multiplex quantitative}

\section{Empirical Analysis}

\subsection{Data}

\begin{table}
\caption{\label{tab:dataSources} Data Sources}
\begin{footnotesize}
\begin{tabular}{llll}
\hline
City & Subject & Data Source & Coverage/Use \\
\hline \hline
Vancouver & Building Permits by  & City of Vancouver & 2019-2025 (laneway and duplex era)\\
& &  & 2024-2025 (laneway, duplex, and multiplex era)\\
Vancouver & Zoning & City of Vancouver & Current R1-1 zones  \\
Vancouver & Transactions & B.C. Assessment & 2017-2019 for baseline price and elasticities \\
Vancouver & Home and lot Characteristics & B.C. Assessment & All \\
\hline
Portland & Building permit  & City of Portland & 2022-2025 (RIP-2 era)\\
Portland & Zoning & City of Portland & RIP-impacted zones\\
Portland & Transactions & Attom & 2018-2021 \\
Portland & Home and lot Characteristics & Attom\\
\hline
Minneapolis & Building permit  & City of Minneapolis & 2020-2025 (post-policy)\\
Minneapolis & Zoning & City of Minneapolis & Interior 1,2, and 3 zones\\
Minneapolis & Transactions & Attom & 2016-2020 \\
Minneapolis & Home and lot Characteristics & Attom\\
\hline
\end{tabular}
\end{footnotesize}
\end{table}

\subsection{Vancouver}

\subsubsection{Sorting in Vancouver single family zones}

A dominant feature of the Vancouver housing market is the role of location. In particular, locations further west have higher value per square foot. \textcite{SanghoonDaryl} show that being just west of Ontario Street, the traditional delineator of the East versus West Sides of Vancouver is associated with a significant difference in price. Figure \ref{fig:WestOld} plots the residuals of a regression of log sale price on log square footage and indicators for age and year sold for single family homes and duplexes with effective ages (based on permitted work) 25 years or less in Vancouver, 2012-2014, well before new duplexes became generically legal in R1-1 zones. The residuals are plotted spatially. This figure shows a very strong East-West divide. Figure \ref{fig:WestNew} plots the same for the period 2022-2024, there remains a divide, but much weaker. 

\begin{figure}
	\caption{\label{fig:WestOld} Price per square foot for single family homes and duplexes in Vancouver, 2012-2014. Residuals from a regression of price of homes on standard (30-36' wide by 100-140' deep) lots on log square footage and indicators for age and year sold.}
\includegraphics[width=\textwidth]{residualsMap2012to2014.png}
\end{figure}

\begin{figure}
	\caption{\label{fig:WestNew} Price per square foot for single family homes and duplexes in Vancouver, 2022-2024. Residuals from a regression of price of homes on standard (30-36' wide by 100-140' deep) lots on log square footage and indicators for age and year sold.}
\includegraphics[width=\textwidth]{residualsMap2022to2024.png}
\end{figure}

Table \ref{tab:regSummariesVancouver} considers different spatial explanations for price per square and elasticity of price with respect to square foot for different levels of spatial aggregation. The unit of observation is a housing transaction between the intermediate years 2017-2019, sales are confined to homes less than 25 years old to abstract from tear-down value. The top panel of Table \ref{tab:regSummariesVancouver} shows that the East/West divide explains over half of residual price variation conditional on square footage. A continuous longitude variable increments adjusted $R^{2}$ only a small amount, and dummies for census tract or broader neighbourhood add little further explanatory power. Thus the East/West divide explains the ``addition'' effect well. The bottom panel of Table \ref{tab:regSummariesVancouver} shows that the same is true for the ``division'' effect. There adjusted $R^{2}$s are presented for regressions with interactions between log square footage and the geography in question. Again, the East/West divide does most of the geographic work.

\begin{table}
	\caption{\label{tab:regSummariesVancouver} Adjusted $R^{2}$ for regressions of log price per square foot (top panel) and location additively (top panel) and interactively (capturing geographic heterogeneity in the elasticity of price with respect to square feet, bottom panel). Single and duplex sales in R1-1 zones 2017-2019. Sales confined to properties 25 years old or less. In the top panel, Longitude x Distance includes longitude, distance to downtown, and their interaction. }
	% latex table generated in R 4.5.2 by xtable 1.8-4 package
% Wed Feb 11 17:07:49 2026
\begin{tabular}{lr}
  \hline
Specification & Adjusted\_R2 \\ 
  \hline
Baseline & 0.51 \\ 
  East Dummy & 0.79 \\ 
  Longitude  & 0.78 \\ 
  Neighbourhood FE & 0.80 \\ 
  Census Tract FE & 0.82 \\ 
   \hline
\end{tabular}


	\par\bigskip\par

	% latex table generated in R 4.5.2 by xtable 1.8-4 package
% Fri Feb 13 16:47:24 2026
\begin{tabular}{lr}
  \hline
Specification & Adjusted\_R2 \\ 
  \hline
Baseline & 0.51 \\ 
  East Dummy Interaction & 0.77 \\ 
  Longitude Interaction & 0.78 \\ 
  Neighbourhood Interaction & 0.81 \\ 
  Census Tract Interaction & 0.82 \\ 
   \hline
\end{tabular}

\end{table}

Table \ref{tab:vancouverEastPPSF} presents the East/West regressions described above. We see that log price per square foot is significantly less on average for East Side properties, but also that the elasticity of price with respect to square footage is significantly less there. 

\begin{table}
	\caption{\label{tab:vancouverEastPPSF} Regressions of log price per square foot on log square footage, an indicator for being on the East Side, and their interaction. Single and duplex sales in R1-1 zones 2017-2019. Sales confined to properties 25 years old or less}
	
\begingroup
\centering
\begin{tabular}{lcc}
   \tabularnewline \midrule \midrule
   Dependent Variable: & \multicolumn{2}{c}{log(ppsf)}\\
   Model:                   & (1)             & (2)\\  
   \midrule
   \emph{Variables}\\
   lsqft                    & -0.2798$^{***}$ & -0.1496$^{***}$\\   
                            & (0.0166)        & (0.0282)\\   
   eastTRUE                 & -0.4198$^{***}$ & 1.097$^{***}$\\   
                            & (0.0100)        & (0.2673)\\   
   lsqft $\times$ eastTRUE  &                 & -0.1966$^{***}$\\   
                            &                 & (0.0346)\\   
   \midrule
   \emph{Fixed-effects}\\
   saleYear                 & Yes             & Yes\\  
   age                      & Yes             & Yes\\  
   \midrule
   \emph{Fit statistics}\\
   Observations             & 1,592           & 1,592\\  
   R$^2$                    & 0.57012         & 0.57881\\  
   Within R$^2$             & 0.53592         & 0.54530\\  
   \midrule \midrule
   \multicolumn{3}{l}{\emph{IID standard-errors in parentheses}}\\
   \multicolumn{3}{l}{\emph{Signif. Codes: ***: 0.01, **: 0.05, *: 0.1}}\\
\end{tabular}
\par\endgroup



\end{table}

Table \ref{tab:vancouverFittedValues} presents out-of-sample fitted values for different homes on a 33x110 foot lot, with a laneway home at .16 FSR, a single-family stand-alone and duplex at .7 FSR and .7/2 FSR, respectively, and a fourplex unit at 1/4 FSR. We find that in all cases, the West Side values are greater. However, the difference in total price (units times price per unit) for a duplex and a fourplex versus a single family home are greater on the East Side than on the West Side. This rationalizes the differential patterns of duplex and fourplex housing we see in Vancouver, with more infill on the East Side than the West Side, even though prices are higher on the West Side. 

\begin{table}
	\caption{\label{tab:vancouverFittedValues} Out-of-sample fitted values for different home types on a 33x110 foot lot, with a laneway home at .16 FSR, a single-family stand-alone and duplex at .7 FSR and .7/2 FSR, respectively, and a fourplex unit at 1/4 FSR. No delta method or similar correction for exponentiaing estimated price per square foot. \textbf{Takeaway: Duplex for sure negatively sorted, fourplex closer call. Laneway ambiguous, hard to quantify expected privacy effect}}
	% latex table generated in R 4.5.2 by xtable 1.8-4 package
% Fri Feb 13 16:47:24 2026
\begin{tabular}{llrrrrr}
  \hline
type & east & sqft & fittedPPSF & units & totalPrice & diffSingle \\ 
  \hline
laneway & FALSE & 580.80 & 1411.87 & 1.00 & 820014.10 &  \\ 
  single & FALSE & 2541.00 & 1132.22 & 1.00 & 2876971.02 &  \\ 
  duplex & FALSE & 1270.50 & 1255.90 & 2.00 & 3191241.90 & 314270.88 \\ 
  quadplex & FALSE & 907.50 & 1320.71 & 4.00 & 4794177.30 & 1917206.28 \\ 
  laneway & TRUE & 580.80 & 1209.94 & 1.00 & 702733.15 &  \\ 
  single & TRUE & 2541.00 & 725.94 & 1.00 & 1844613.54 &  \\ 
  duplex & TRUE & 1270.50 & 922.78 & 2.00 & 2344783.98 & 500170.44 \\ 
  quadplex & TRUE & 907.50 & 1036.76 & 4.00 & 3763438.80 & 1918825.26 \\ 
   \hline
\end{tabular}

\end{table}

\subsubsection{Laneway homes}

Table \ref{tab:vancouverFittedValues} shows that laneway homes if they were freehold structures would have fitted prices of \$820,000 on the West Side and \$702,000 on the East Side. Given the West Side's greater willingness to pay for larger and single-family homes than the East Side's however, we might expect an offsetting effect of a greater willingness to pay to avoid the congestion of a laneway home on the West Side. These offsetting effects are consistent with a statistically significant but economically modest correlation of .06 (across census tract means) between an East dummy and the presence of a laneway among new or substantially improved single family homes in R1-1 zones between 2019 and 2025. Figure \ref{fig:lanewayMap} plots the location of these permits. The mean laneway share is 39\% on the West Side and 46\% on the East Side. Illustrating the trend, the shares are 71\% West and 64\% East in 2024-2025. The convergence in laneway shares matches general convergence in single family prices, with the interquartile range falling from 1.93 in 2016 to roughly 1.6 in 2022-2024. 

\begin{figure}
	\caption{\label{fig:lanewayMap} Latitude and longitude of single family home permits that were significant additions or alterations or new construction between 2017 and 2025. Colors of points determined by the presence or absence of a laneway home within 20 meters of the permitted single family home.}
	\includegraphics[width=\textwidth]{lanewayMap.png}
\end{figure}

\subsubsection{Duplexes}

Table \ref{tab:vancouverFittedValues} shows that the difference-in-differences in profitability between a pair of duplexes and a single family home is greater on the East Side than the West Side by a large margin. We see a commensurately visible difference in propensities in Figure \ref{fig:duplexMap}, which plots the location of duplex permits between 2019 and 2023, after the allowance of duplexes city-wide and before multiplexes started to be permitted. The visual difference corresponds to a correlation of .18 between duplex and an indicator for East Side location. The duplex share of permits 2019-2023 is 20\% on the East and 39\% on the West Side.

\begin{figure}
	\caption{\label{fig:duplexMap} Latitude and longitude of duplex and non-duplex (single-family) permits between 2019 and 2023.}
	\includegraphics[width=\textwidth]{duplexMap.png}
\end{figure}

\subsubsection{Multiplexes}

Table \ref{tab:vancouverFittedValues} shows that the fitted difference in profitability between multiplex and single family is almost equal between the East and West Sides (this is out-of-sample, as the estimation is before multiplexes were permitted). Figure \ref{fig:multiMap} plots the location of multiplex permits and other (single family or duplex homes) between 2024 and 2025, after the allowance of multiplexes city-wide. The visual difference corresponds to a tract-level correlation of an intermediate correlation of .15 between multiplex and an indicator for East Side location among standard lots and .23 for wide (40' or more) lots. The multiplex share of 2024-2025 permits on standard lots is 13\% on the West Side and 25\% on the East Side. On wide lots, the shares are 25\% and 49\%, respectively.

\begin{figure}
	\caption{\label{fig:multiMap} Latitude and longitude of multiplex and non-multiplex (single-family or duplex) permits between 2024 and 2025. Wide (40' or greater) lots.}
	\includegraphics[width=\textwidth]{multiMap.png}
\end{figure}


\subsection{Minneapolis}

In Minneapolis and Portland, using Attom transaction data, we estimate the mean price per square foot and elasticity of price per square foot in home square footage for homes in formerly single family zones in the years leading up to the upzoning. Zip-code specific elasticities come from a regression of the form, where $z$ represents a Zip. $P$ is price.

\begin{equation}
	\label{eq:attomReg}
	\ln \frac{P_{izt}}{\text{sqft}_{izt}} = \beta_{0z} + \beta_{1z} \ln \text{sqft}_{izt} + \gamma X_{izt} + \delta_t + \epsilon_{izt}
\end{equation}

We combine the estimated elasticities $\beta{1}$ and simply calculated median price per square foot by zip code, and then merge at the zip code level with each city's permit data to find the fraction of permits in upzoned formerly single-family zones that are for duplexes or triplexes (or more in the case of Portland). We then calculate correlations between the estimated elasticities, price per square foot, median income, and the propensity to build duplexes or triplexes. For Minneapolis, Table \ref{tab:MinneapolisCorrelations} shows this for all 34 affected Zip Codes/ZCTAs (top panel), for those inside the interior-3 zone, in which density bonuses of .2/.5 and .3/.5 floor area ratios were granted. Pursuant to our theoretical discussion, we expect a more positive correlation between price level and propensity to build plexes where the addition effect is larger (or existent, zone 3) than where the division effect is the only effect (zones 1 and 2). That theoretical result is supported by comparing the middle to the bottom panel, with the caveat that the number of zip codes is very small. Overall, in Minneapolis, we identify 509 permits on affected lots of which 198 are multiplex applications. 

\begin{table}
	\caption{\label{tab:MinneapolisCorrelations} Correlations between price level, elasticity of price with respect to square footage, median income, and propensity to build duplexes or triplexes in Minneapolis. All affected Zips (top panel), Interior-3 zone (middle panel, addition and division effects apply), and Interior-1 and 2 zones (bottom panel, division effect only). \textbf{Takeaway: positive sorting only where there is an addition effect.}}

	\textbf{All affected Zips}

\begin{tabular}{rrrrr}
  \hline
 & elasticity & price\_level & propensity & medianIncome \\ 
  \hline
elasticity & 1.00 & 0.59 & 0.28 & 0.30 \\ 
  price\_level & 0.59 & 1.00 & 0.03 & 0.69 \\ 
  propensity & 0.28 & 0.03 & 1.00 & -0.14 \\ 
  medianIncome & 0.30 & 0.69 & -0.14 & 1.00 \\ 
   \hline
\end{tabular}

\par\bigskip\par

\textbf{Interior-3 zone}\par\bigskip\par

\begin{tabular}{rrrrr}
  \hline
 & elasticity & price\_level & propensity & medianIncome \\ 
  \hline
elasticity & 1.00 & 0.61 & 0.32 & 0.31 \\ 
  price\_level & 0.61 & 1.00 & 0.24 & 0.82 \\ 
  propensity & 0.32 & 0.24 & 1.00 & 0.01 \\ 
  medianIncome & 0.31 & 0.82 & 0.01 & 1.00 \\ 
   \hline
\end{tabular}
\par\bigskip\par

\textbf{Interior-1 and 2 zones}\par\bigskip\par
\begin{tabular}{rrrrr}
  \hline
 & elasticity & price\_level & propensity & medianIncome \\ 
  \hline
elasticity & 1.00 & 0.58 & 0.36 & 0.29 \\ 
  price\_level & 0.58 & 1.00 & -0.05 & 0.64 \\ 
  propensity & 0.36 & -0.05 & 1.00 & -0.30 \\ 
  medianIncome & 0.29 & 0.64 & -0.30 & 1.00 \\ 
   \hline
\end{tabular}
\end{table}

\subsection{Portland}

\textbf{In Portland, almost all of the infill uptake is in Zones 2.5 and 5, close to the urban core. Plausibly, FAR limits were not binding on large lots in the outer zones. Among building permits by zone, the mean infill propensity across zip codes (one mean per zip code) is 26.3\% for R 2.5, 10\% for R5, 5.9\% for R7, 2.9\% for R10, and 0\% for R20. Across all zip codes-zoning code pairs for which we compute infill propensities, mean prices, and elasticities of price in square foot, the correlation between uptake and price level is -.2. Across the only 5 zip codes in R2.5, we find a correlation of .71, and across the 9 zip codes in R5, the correlation is -.85. These samples are too small to draw conclusions, but the overall negative correlation reflects differences driven by an apparent negative correlation between lot size and price per square foot (driven by a higher price per square foot in R5 than R2.5). }

\section{Conclusion}

We present a simple model of infill housing benefits that distinguishes between the benefits of additional square footage and the ability to divide square footage. We find elasticities of price with respect to square footage generally significantly below one in Vancouver and Portland, consistent with strong demand for the division effect. This model does a fair but incomplete job of explaining the seeming puzzle of stronger uptake of infill where real estate is less valuable, so that the addition effect is weaker.

In standard urban models, there is uncerainty over the correlation between price per square foot and incomes (and hence, presumably, demand for division). This is because both locational quality (e.g. time saved commuting and public goods) and space are normal goods. When zoning binds so households are largely forced to overconsume space, households have limited scope to sort on demand for space, and hence sorting is strongly positive on income across locations.

In Vancouver, there are generally negative correlations between price per square foot and infill activity across different specifications of geography. There is relatively little difference, particularly in recent years in uptake of the laneway home option among single family homes, consistent with offsetting price per square foot effects (larger on the West Side) and privacy costs and small-unit benefit effects (smaller and larger respectively on the East Side). Duplex units involve only a division effect, with no addition benefit (and indeed lost laneway square footage). Consistent with our model, duplex units are roughly twice as popular on the cheaper (relatively!) and less affluent East Side than the West Side. Multiplexes have both addition and division effects. We find on standard lots that these effects are roughly a wash in terms of the profitability (on the revenue side only) on the East versus West Sides. As with duplexes, multiplexes are roughly twice as popular on the East as the West Side. Our model makes this result less puzzling, but does not quantitatively explain the effect as currently calibrated.

In Minneapolis, we find again a negative correlation between infill activity and price per square foot across all affected Zip Codes. However, in the relatively inner-city area where both addition and division effects apply, the correlation is positive, versus negative where there is only a division effect in outlying areas. Similarly in Portland, a strong negative correlation between price per square foot and infill activity is driven in large part by differences in lot sizes and the extent to which older floor space ratios were binding. In the active inner zones, the correlation is ambiguous within the small and larger-lot zones.

\begin{itemize}
	\item Division effects only: negative sorting (Vancouver duplexes)
	\item Addition effects only: not observed
	\item Mixed examples: mixed results (Interior-3 in Minneapolis positive, multiplexes in Vancouver negative, R2.5 in Portland positive, R5 in Portland negative, Vancouver laneways mostly negative)

\end{itemize}

\printbibliography
\end{document}


