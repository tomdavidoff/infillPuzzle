\documentclass[12pt]{article}

\usepackage{amsmath}
\usepackage[margin=1in]{geometry}
\usepackage{graphicx}
\usepackage{setspace}

\begin{document}

\title{Where does the infill housing go?}
\author{Thomas Davidoff\thanks{thomas.davidoff@sauder.ubc.ca} and Tsur Somerville\thanks{tsur.somerville@sauder.ubc.ca}\\Sauder School of Business, University of British Columbia}
\maketitle

\begin{abstract}
  Big house for rich people not the poors.
\end{abstract}

\section{Model}

Consider a developer considering the intensity of development on a parcel of land with area normalized to one unit. Profits are a function of the (continuous, for simplicity) number of units $n$ built on the lot. Zoning regulations impose a schedule of density limits $f(n)$ governing how many square feet $f$ can be built on the lot as a function of $n$. The price $p$ per square foot may also be functions of $n$ and $\theta$, a characteristic of the neighbourhood.


\begin{equation}
  \pi(n) = f(n)\left[p\left(n,\theta\right) - c(n)\right]
\end{equation}

The benefit from incrementing density is:
\begin{equation}
  \label{eq:mb}
  \frac{\partial \pi}{\partial n} = f'(n)\left[p\left(n,\theta\right) - c(n)\right] + f(n)\left[\partial{p\left(n,\theta\right)}{\partial f} - c'(n)\right]
\end{equation}

Optimizing for $n^{\star}$ yields the first order condition:
\begin{equation}
  \label{eq:foc}
  f'(n^{\star})\left[p(n^{\star},\theta) - c(n^{\star})\right] + f(n^{\star})\left[\frac{\partial (n^{\star},\theta)}{\partial n} - c'(n^{\star})\right] = 0
\end{equation}

Implicitly differentiating (\ref{eq:foc}) with respect to $\theta$ yields to find the effect of $\theta$ on optimal density:
\begin{equation}
  \frac{d n^{\star}}{d \theta} = -\frac{f'\frac{\partial p}{\partial \theta} + f\frac{\partial^2 p}{\partial n \partial \theta}}{f''\left[p - c\right] + 2f'\left[\frac{\partial p}{\partial n} - c'\right] + f\left[\frac{\partial^2 p}{\partial n^2} - c''\right]}.
\end{equation}

Assuming second order conditions hold, the denominator is negative, and:

\begin{itemize}
  \item Infill (increasing $n$) is more attractive where the price per square foot is greater ($\frac{\partial p}{\partial \theta} > 0$) when $f'$ is larger (i.e., where infill is rewarded with bonus density).
  \item Infill is more attractive where price is declining more rapidly in density ($\frac{\partial^2 p}{\partial n \partial \theta}$ is more positive).
\end{itemize}

If $\theta$ is overall neighbourhood quality, measured as average price per square foot of (assumed constant quality) structures, infill will be more prevalent in expensive neighbourhoods when bonus density is generous. Whether incoe 

\section{Empirical Analysis}

We consider the following densification programs:

\begin{itemize}
  \item Vancouver allowing laneway homes in single family zones (RS) starting in 2009. 


\end{document}
