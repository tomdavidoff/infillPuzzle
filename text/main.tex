\documentclass[12pt]{article}

\usepackage{amsmath}
\usepackage[margin=1in]{geometry}
\usepackage{graphicx}
\usepackage{setspace}

\begin{document}

\title{Where does the infill housing go? Addition versus division}
\author{Thomas Davidoff\thanks{thomas.davidoff@sauder.ubc.ca} and Tsur Somerville\thanks{tsur.somerville@sauder.ubc.ca}\\Sauder School of Business, University of British Columbia}
\maketitle

\begin{abstract}

Jurisdictions across Canada and the U.S. have recently allowed ``missing middle'' infill housing such as accessory dwelling units and multiplexes in residential zones formerly restricted to single-family use. Some studies have documented that infill housing is less prevalent in more expensive neighbourhoods. This is puzzling in that the right to add density should be more valuable in neighbourhoods where prices are higher. To resolve this seeming contradiction, we develop a parsimonious model emphasizing that different implementations of missing middle zoning feature different mixes of ``addition'' (adding to allowable square footage) and ``division'' (subdividing a fixed amount of square footage into multiple units). Addition is more valuable where price per square foot of built space is greatest. Division is more valuable where the elasticity of price in square footage is less positive. Infill housing will thus be less prevalent in pricier neighbourhoods when the upzoning features relatively little addition and more division if (wealthier) households with greater willingness to pay for space  are concentrated in expensive price-per-square-foot neighbourhoods. When that sorting is stronger, the prevalence of infill will be more negatively associated with price growth. 

Housing transactions and permitting data in upzoned neighbourhoods of Vancouver, Portland, and Minneapolis generally support these conclusions from our simple model of optimal infill choice. In Vancouver, where there is a very high degree of sorting, the correlation between neighbourhood price-per-square-foot levels and the missing middle share of permits has been particularly negative. In Minneapolis, the relationship between infill propensity and price is negative in modestly upzoned areas, but positive where permitted intensity was more strongly increased for multiplexes.
  
\end{abstract}

\end{document}

\section{Model}

Consider a developer considering the intensity of development on a parcel of land with area normalized to one unit. Profits are a function of the (continuous, for simplicity) number of units $n$ built on the lot. Zoning regulations impose a schedule of density limits $f(n)$ governing how many square feet $f$ can be built on the lot as a function of $n$. The price $p$ per square foot may also be functions of $n$ and $\theta$, a characteristic of the neighbourhood.


\begin{equation}
  \pi(n) = f(n)\left[p\left(n,\theta\right) - c(n)\right]
\end{equation}

The benefit from incrementing density is:
\begin{equation}
  \label{eq:mb}
  \frac{\partial \pi}{\partial n} = f'(n)\left[p\left(n,\theta\right) - c(n)\right] + f(n)\left[\partial{p\left(n,\theta\right)}{\partial f} - c'(n)\right]
\end{equation}

Optimizing for $n^{\star}$ yields the first order condition:
\begin{equation}
  \label{eq:foc}
  f'(n^{\star})\left[p(n^{\star},\theta) - c(n^{\star})\right] + f(n^{\star})\left[\frac{\partial (n^{\star},\theta)}{\partial n} - c'(n^{\star})\right] = 0
\end{equation}

Implicitly differentiating (\ref{eq:foc}) with respect to $\theta$ yields to find the effect of $\theta$ on optimal density:
\begin{equation}
  \frac{d n^{\star}}{d \theta} = -\frac{f'\frac{\partial p}{\partial \theta} + f\frac{\partial^2 p}{\partial n \partial \theta}}{f''\left[p - c\right] + 2f'\left[\frac{\partial p}{\partial n} - c'\right] + f\left[\frac{\partial^2 p}{\partial n^2} - c''\right]}.
\end{equation}

Assuming second order conditions hold, the denominator is negative, and:

\begin{itemize}
  \item Infill (increasing $n$) is more attractive where the price per square foot is greater ($\frac{\partial p}{\partial \theta} > 0$) when $f'$ is larger (i.e., where infill is rewarded with bonus density).
  \item Infill is more attractive where price is declining more rapidly in density ($\frac{\partial^2 p}{\partial n \partial \theta}$ is more positive).
\end{itemize}

If $\theta$ is overall neighbourhood quality, measured as average price per square foot of (assumed constant quality) structures, infill will be more prevalent in expensive neighbourhoods when bonus density is generous. Whether incoe 

\section{Institutional setting}

In this paper, we assess patterns of redevelopment of single-family sites to modestly higher density in Minneapolis, MN; Portland, OR; and Vancouver, BC, Canada. Below, we describe the evolution of the relevant policies and their details in each of these three cities.

\textit{Minneapolis, MN USA}

Of the cities included here, Minneapolis was the first to move away from single-family zoning, with the City Council voting in December 2018 to allow more than one unit on all single-family lots. The changes came into effect in January 2020 following approval by the region’s Metropolitan Council in 2019. The rezoning divides the city into three major regions, with additional distinct policies for transit corridors. These corridors were rezoned to allow 3–6 storey buildings, with 10–30 storeys permitted close to transit stations. Additionally, in May 2021, off-street parking minimums were eliminated.

Our interest is in the three largest regions of the city: “Interior-1,” “Interior-2,” and “Interior-3,” where the rezoning permitted up to three units on all previously single-family-zoned lots. The largest land area is in Interior-1, which consists of neighborhoods most distant from the city’s core. In this area, the floor space ratio (FSR) remained unchanged at 0.5, with a maximum height of 2.5 storeys or 28 feet (8.5 meters). In Interior-2, the ring around the core, the same restrictions apply. Only in Interior-3—the single-family areas immediately south and northeast (across the Mississippi River) of the downtown core—did the rezoning increase FSR, from 0.5 to 0.6 for duplexes and 0.7 for triplexes. Triplexes also received a relaxation of the height restriction, to three storeys or 42 feet (12.8 meters). The standard residential lot size in Minneapolis is 5,000 sf, with newer single-family units typically maximizing the allowed buildable area.

\textit{Portland, OR USA}

Broad upzoning in Portland was triggered by the Oregon legislature’s passage of House Bill 2001 in July 2019. This bill effectively banned exclusive single-family zoning and required jurisdictions with over 25,000 residents, among other changes, to allow up to four units per lot. Cities were given until June 2022 to update their zoning codes to meet the bill’s requirements.

The City of Portland enacted its first round of reforms, the Residential Infill Project (RIP-1), effective in August 2021. This first of two rounds applied to zones nearer the city centre with smaller lots (land use zones R2.5, R5, and R7).\footnote{These zones reflect allowed lot sizes: R2.5 allows 1 lot per 2,500 sq ft; R5 allows 1 lot per 5,000 sq ft; and R7 allows 1 lot per 7,000 sq ft.} Allowed FSR varied by zone and by the number of units built per lot, with an additional 0.1 FSR for each added unit above the single-family maximum of 0.7, 0.5, and 0.4 in zones R2.5, R5, and R7, respectively.\footnote{For a 2-, 3-, or 4-unit development, the maximum FAR would be 0.8, 0.9, and 1.0 in the R2.5 zone, respectively; 0.6, 0.7, and 0.8 in the R5 zone; and 0.5, 0.6, and 0.7 in the R7 zone.}

The City enacted a second round of reforms (RIP-2) in June 2022. This second round extended the RIP-1 changes to the lower-density, larger-lot zones (R10 and R20, for 10,000 and 20,000 sq ft lots, respectively), which tended to be further from the city’s urban core.\footnote{The shares of single-dwelling land by zone are roughly R2.5 – 7%, R5 – 41%, R7 – 18%, R10 – 18%, and R20 – 5%; see \cite{DongHansz_2019}.} Additional changes under RIP-2 included the ability to subdivide lots and create townhouse-style, side-by-side, fee-simple units across all zones, as well as the extension of accessory dwelling unit (ADU) permissions into the R10 and R20 zones. Overall, RIP-2 extended RIP-1 to the remaining single-family zones, while permitting more flexibility in structure types and options for fee-simple, rather than condominium or strata, ownership.

\textit{Vancouver, BC Canada}

In Vancouver, increases in allowed densities on single-family lots occurred in two rounds. The first was the September 2018 approval of duplexes, with up to two additional rental basement suites (one per unit) permitted on nearly all single-family lots.\footnote{The primary excluded single-family area is the heritage district in the Shaughnessy neighbourhood, comprising very large houses on large lots.} The inclusion of rental basement suites is consistent with longstanding single-family zoning, under which nearly all lots are allowed a single main structure with a rental basement suite and, where a lane is present, a rental infill laneway house or ADU.

Duplexes and single-family units faced the same maximum FSR, between 0.6 and 0.7. These FSR restrictions did not apply to laneway units, which were permitted an additional 0.16 FSR. Duplexes could not include laneway units. Thus, a single-family property could contain one owned unit and two rental units, while a duplex could contain two owned units, each with its own rental suite. Total built FSR could therefore be higher for a single-family property with a laneway unit than for a duplex.

Following the Province of British Columbia’s elimination of exclusive single-family zoning in November 2023, the City of Vancouver extended multiplex permissions to all residential zones. At the same time, the maximum allowed size for ADU units was increased from 0.16 to 0.25 FSR. The number of permitted units varies between three and six (and up to eight for purpose-built rental), depending on lot area and frontage. A standard 33’ × 122’ lot would allow three units. To allow four units, lot area would have to exceed 4,984 sq ft or frontage would have to exceed 43.6’.

With the 2023 changes, maximum allowed FSR depended on several factors. Single-family homes and duplexes were permitted maximum FSRs of 0.6 and 0.70, respectively, with an additional 0.25 FSR allowed for an ADU on single-family lots. For multiplexes (three or more units), the default maximum was 0.7, with 1.0 FSR attainable if one of the following applied: (i) all units are secured-rental housing; (ii) one unit is designated as “below-market” homeownership; or (iii) the developer pays a density bonus, which can vary from \$3.00 to \$140.00 per square foot of additional density depending on lot size and location. For lots of sufficient size, 7–8 rental units are also possible at an FSR of 1.0.


\section{Empirical Analysis}

For each dated permitting event in each city, we calculate profits from the available choices. For R zones in Vancouver, the profit from a single family home is approximated as the profit from a single family home with a laneway. In some neighbourhoods, takeup... laneway home.



We consider the following densification programs:

\begin{itemize}
  \item Vancouver allowing laneway homes in single family zones (RS) starting in 2009. 


\end{document}
